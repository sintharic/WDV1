\chapter{Elementare Datenverarbeitung}

\section{Integrieren}

Es liegen Daten einer Funktion $f(x)$ an diskreten Stütztstellen im Format $x_n, f_n$ vor und es soll das Integral $I=\int_{x_0}^{x_N}f(x)\mathrm{d}x$ näherungsweise bestimmt werden oder aber die Stammfunktion $F(x)$ an den vorliegenden Stützstellen.
%
Die Integrationskonstante wird zunächst so festgelegt, dass $F(x_n) = 0$. 
%
Zur Lösung der Aufgabe bietet sich oft die Trapezregel an, siehe auch englisches Wiki, mit der das Integral zwischen $x_{n}$ und $x_{n+1}$ wiefolgt abgeschätzt wird:
\begin{eqnarray}
\int_{x_n}^{x_{n+1}} f(x) dx & = & (x_{n+1}-x_{n}) \langle f \rangle_n  \nonumber\\
& \approx & \Delta x_n \frac{f(x_{n})+f(x_{n+1})}{2},
\end{eqnarray}
Hierbei ist $\Delta x_n = x_{n+1}-x_n$ und
 $\langle f \rangle_n$ ist, gemäß dem Mittelwertsatz der Integralrechnung, der Mittelwert der Funktion $f(x)$ auf dem Intervall $[x_n,x_{n+1}]$. 
%

\textbf{Detailinformation:} 
Der führende  Fehler der Integration in diesem Schema ist in der Regel $f''(x_n) \Delta x_n^3$.
%
Über das Integrationsgebiet akkumuliert er sich zu: (mittlere Krümmung mal $N$) mal $\Delta x^3$, was letztendlich proportional zu $1/N^2$ bzw. $\Delta x^2$ ist. 
%
Man sagt auch der (erwartete) Fehler hat die Ordnung $1/N^2$ oder einfach $\mathcal{O}(1/N^2)$. 
%
Eine Verdoppelung der Anzahl der Stützstellen bewirkt eine Vervierfachung der Genauigkeit. 


In manchen Fällen divergiert eine Funktion, wie z.B. $f(x)=1/\sqrt{x}$ bei $x_0 = 0$. 
%
Aus Python-Sicht müsste dann ein exception handling definiert werden, auf das wir u.U. später zurückkommen.
%
Liegt die Funktion jedoch analytisch vor, hilft oft eine Näherung der Form
\begin{eqnarray}
\langle f \rangle_n & \approx & f(x_{n+1/2}) \textrm{ mit } x_{n+1/2} = (x_n+x_{n+1})/2\\
\Rightarrow \int_{x_0}^{x_N} f(x) \mathrm{d}x & \approx & \Delta x \sum_{n=0}^{N-1} f(x_{n+1/2})
\label{eq:midpointRuleIntegration} % nicht korrekter Name!!!
\end{eqnarray}
weiter, wobei hier fairerweise gesagt werden muss, dass in solchen Fällen die Verwendung einer konstanten Schrittweite $\Delta x$ nicht unbedingt optimal ist.
%
Dabei sollte aber, was oft  verschwiegen wird, nicht nur der Zeitaufwand des Computers sondern auch der des Programmierers berücksichtigt werden, sodass die in Gl.~\eqref{eq:midpointRuleIntegration} beschriebene Näherung eben doch optimal sein kann. 

Als Beispiel nehmen wir die Funktion $f(x) = \sqrt{x}\sin(x)$ auf dem Intervall $[0,4\pi]$ unter Verwendung von zunächst $N = 40$ Stützstellen. Dies entspricht einem $\Delta x$ von $\pi/10$, also quasi $18^\circ$, was einiges größer ist als der Bereich  von $\vert \varphi \vert < 5^\circ$, in dem die Kleinwinkelnäherung von $\sin\varphi \approx \varphi$ gerne als gültig angegeben wird. 
%


\lstinputlisting[language=Python, label=lst:trapezoid, caption=Traditionelle Integration gemäß der Trapezregel]{ElementarCode/code1.py}
% \begin{lstlisting}[language=Python, label=lst:trapezoid, caption=Traditionelle Integration gemäß der Trapezregel]
% import numpy as np
% 
% xMin = 0
% xMax = 4*np.pi
% N = 40
% 
% def myFunction(x):
%   return np.sqrt(x)*np.sin(x)  # keine Klammern im return statement
% 
% dx = (xMax-xMin)/N
% x = np.linspace(xMin,xMax,N+1)
% f = []
% F = [] # Gross- und Kleinschreibung wird unterschieden
% 
% f.append(myFunction(xMin))
% F.append(0)
% print(x[0], f[0], F[0])
% 
% for i in range(N):
%   j = i+1
%   f.append(myFunction(x[j]))
%   dF = dx*(f[i]+f[j])/2
%   F.append(F[i]+dF)
%   print(x[j], f[j], F[j])
% \end{lstlisting}

Als nächstes: numerische Stammfunktion mit numpy Paket.

Es ist Zeit zu gestehen, dass wir im ``echten Leben'' dieses konkrete  Beispiel, in dem die $x_n,f_n$ nicht als Messwerte vorliegen sondern als einfache Funktion, mit Hilfe von \texttt{wolframalpha} (wa) wiefolgt lösen würden:\vspace*{2mm}\\
\texttt{wa> N[integrate sqrt(x)*sin(x) from x = 0 to 4*pi,10]} \vspace*{2mm}\\
Hierbei gibt die Zahl \texttt{10} die Anzahl der gültigen \texttt{N}achkommastellen an. 

Erste Verwendung von \texttt{sympy}


\section{Differenzieren}

\section{Statistische Datenanalyse}
