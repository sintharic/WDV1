\chapter{Vorbemerkungen zu Kurs und Skript}

In diesem Kurs soll Ihnen beigebracht werden, Computer zur Unterstützung diverser anderer Fächer zu verwenden.
%
Als die dazu beste Programmiersprache hat sich (momentan) Python herauskristallisiert, auf das sich dieser Kurs fokussiert. 
%
Ob Python auch in 30~Jahren noch das Ma{\ss} der Dinge ist, ist schwer vorherzusagen.
%
Im Jahr 2022 ist es jedoch auf dem aufsteigenden Ast und ein Ende seines Siegeszuges ist nicht abzusehen.

Für den Kurs haben wir einen ``ganzheitlichen'' Ansatz gewählt, der von üblichen Programmier-Kursen abweicht.
%
Sie lernen also im übertragenen Sinne zunächst ganze Sätze ohne mit vielen grammatikalischen Begriffen konfrontiert zu werden. 
%
Damit sollen Sie schnell in die Lage versetzt werden, Python gewinnbringend zur Unterstützung in anderen Kursen einzusetzen. 
%
So werden kontinuierlich durch das Skript, Funktionen aus den Paketen \texttt{numpy} oder \texttt{scipy} eingeführt, mit dem sich viele Aufgaben in einer Zeile lösen lassen.
%
Es gibt aber keine eigenständige Kapitel über diese Pakete. 
%
Gleichzeitig wird an einigen Beispielen auch Hintergrund z.B. zu Numerik oder Statistik formlos gelehrt oder es wird aufgezeigt, wie die Probleme mit herkömmlichen Programmen, allerdings in Python Syntax, gelöst würden.
%
Am Ende von WDV-I sollten Sie dann in der Lage sein, echte Python Bücher (quer) zu lesen, aber auch schnell andere Programmiersprachen wie z.B. C++ zu erlernen. 

In den Hausaufgaben, werden Sie selbst kleine Python Programme entwickeln und diese in elektronischer Form abgeben.
%
Vorlesung, Skript und Anweisungen in Hausaufgaben sollten in aller Regel als Material genügen, ansonsten gibt es google und auch Literaturangaben, die im Skript eingestreut werden.

Als Leistungskontrollen wird es zwei Testate geben, deren Bestehen Ihnen erlaubt an der finalen Leistungskontrolle, der Klausur teilzunehmen.
%
Weder bei Testaten noch Klausuren sind Computer erlaubt.
%
Das klingt paradox, aber es wäre sonst zu einfach zu mogeln.
%
Leistungskontrollen werden vermutlich aus zwei, manchmal drei Teilen bestehen:
\begin{enumerate}
\item Ein Teil, in dem Ihnen Code gegeben wird und Sie vorhersagen müssen, was bei dessen Ausführung passiert.
\item Ein Teil, in dem Sie selbst Python Code schreiben. 
\item Unter Umständen werden Sie auch aufgefordert, Fehler in gegebenem Code zu markieren. 
\end{enumerate}

\textbf{Dieses Skript ist kein Lehrbuch.} Details zur Syntax und andere Regeln werden in der Vorlesung und den Übungen erklärt oder können in den angegebenen Referenzen nachgeschlagen werden. Dieses Skript ist eher eine Ansammlungen von kommentierten, kleinen Codes. In der Regel wird auch kein Output der Programme gezeigt. Den können und sollen Sie nach einmausen von Code aus dem PDF selbst erzeugen. 

\subsubsection{Ablauf einer Lehreinheit:} Eine Lehreinheit besteht aus 3 Lehrstunden je 45~min. In der 1. Stunde werden die Hausaufgaben besprochen, in der 2. Stunde neue Inhalte vermittelt und die 3. Stunde dient dazu, dass Studierende mit der Lösung des neuen Aufgabenblattes beginnen und dabei Unklarheiten oder andere Schwierigkeiten an die anwesenden HiWis stellen können.

